\documentclass[sigconf]{acmart}
\usepackage{graphicx}
\usepackage{hyperref}
\usepackage{todonotes}

\usepackage{endfloat}
\renewcommand{\efloatseparator}{\mbox{}} % no new page between figures

\usepackage{booktabs} % For formal tables

\settopmatter{printacmref=false} % Removes citation information below abstract
\renewcommand\footnotetextcopyrightpermission[1]{} % removes footnote with conference information in first column
\pagestyle{plain} % removes running headers

\newcommand{\TODO}[1]{\todo[inline]{#1}}
\begin{document}
\title{Netflix use of Big Data Visualization}
\author{Zachary Meier}
\affiliation{%
  \institution{Indiana University}
  \city{Bloomington} 
  \state{Indiana} 
  \postcode{47408}
}
\email{zrmeier@indiana.edu}
% The default list of authors is too long for headers}
\renewcommand{\shortauthors}{Z. Meier et al.}
\begin{abstract}
When you think of Netflix you probably don't think about big data.  You think about streaming videos and being watching the newest series.  How they keep you watching and a happy customer is through their use of big data.  Not only does Netflix use big data for their external customers they use it for their internal customers.  Netflix use of big data analytics has given them strong customer insight and internal applications to help keep their systems up and running and striving for the four nines availability.  The uptime of 99.99\% which they strive for and have tried to uphold very well.
\end{abstract}
\keywords{Big Data, Edge Computing i523}
\maketitle
\section{Introduction}
Netflix was once a small start up company, trying to compete with Blockbuster and Hollywood Video.  It's main goal was to allow customers to get DVD's through the mail without having to go in store.  They expanded their services through distributing an API to other companies to encourage streaming and coming back to their site.  That originally turned out to be a failure, but then they started to see the value in streaming, and started to allocate major resources in that direction.  From there they became a big data giant in the movie streaming business.
\section{Brief History}
There is no doubt you have not heard of Netflix, but their climb to where they are now has been a journey of failure and strife.  Netflix once was a DVD only shipping business, but soon started experimenting with streaming by pushing out their API to other companies to drive business back to them.  However this had been a huge failure, but through that they found value in streaming on their own.  They started building teams and software to create to handle the data they encountered and originally started with their own infrastructure as well to host this service.  Eventually they saw value in the use of micro services and the use of amazon web series as a solid company structure.  Eventually they started living exclusively in the cloud.  With that they faced many challenges but they knew that the pro's out weight the con's and pushed forward into the new frontier of cloud computing and micro service industry.  Netflix now has almost 100 million views in almost every country around the globe.  However, they plan to push for more customers.  They plan to dedicate 1 billion dollars to acquiring new customers.  It is uncertain what the future hold for Netflix but one can bet that it will involve big data, and that is what will drive them forward.\cite{Marr}
\section{The Revolution of the Genre}
If someone wanted to count how many genres were out there for movies, they would be able to count it easily.  However, when Netflix was looking for a way to create better data so they could create better suggestions to retain customers, they revolutionized that system.  Gone are the days of simplistic genres of the past created by Hollywood.  Now are the days of 76,897 to categorize movies according to Netflix. \cite{Madrigal}  Soon this grew into much more.  
\subsection{The Netflix Quantum Theory}
This was a document in which was created to come up with a method of tagging movies with genres.  In this case they were called micro genres.  This set out to distinguish movies from one another while making it easier for a generator to put together movies someone would most likely watch based on the genres provided.  Such as quirky movie or romantic comedy featuring the rock.  Does that actually exist?  If it did, Netflix Quantum Theory Generator would produce something like that to get people to watch it.  This is the basis for their predictive algorithm which they are always fine tuning to provide value to their customers.
\section{Internal Use of Big Data}
Their is no doubt of use for big data to help customers find the movies they would like to watch.  However that is only one side of the issue.  For Netflix to be effective they need to be able to take the data they get back from the customers and turn it into something tangible and usable, so they can keep creating value for their customers.  This is included in their use of content creation, not just the utilization of the license agreement from the movie companies.  They use all of this information to keep track of their ecosystem of micro services as well as see what users are doing and how to better serve them.  Due to their overwhelming size and use of data, Netflix had to get creative and start thinking on better ways to collect and push this data across all their internal teams to produce a company that works in harmony.  
\subsection{Atlas}
Atlas is essentially the monitoring tool that Netflix uses to make their operation work.  They implemented this system back in 2013 and it has been running since. It allows for easy time series queries to beget back valuable data that scales.  They did this by simplifying the structure of the data and the query structures.  In addition most of this data was kept in memory so it as easily accessible so as not have to access a database and slow down the process.  This allows the data to be pulled quickly from many different services that demand the data.  \cite{Netflix}
\subsection{Visualization}
It is one thing to understand the data from a mathematical and theoretical level.  However that only helps those who are easily associated with it. At Netflix, if it is not visualized then it doesn't serve a purpose. Which cover the three ways Netflix views data.
\begin{itemize}
\item Data should be accessible, easy to discover, and easy to process for everyone.
\item Whether your dataset is large or small, being able to visualize it makes it easier to explain.
\item The longer you take to find the data, the less valuable it becomes.\cite{Simon}
\end{itemize}
With these view you can see how Netflix want to accurate rich accurate data.  Not only that but make it better.  Something a normal person can look at, and be able to make a informed decision.  For instance, in the case of choosing a new shows title picture.  They put out different version and see what gets chosen, and all the characteristics of why that particular one was chosen.  Was it based off color, people in it, typography.  They can usually those data points to help pick the best picture to have people watch that show.  That is one of the key drivers for why their business has continued to flourish.  
\subsection{Micro services}
Not only does all of this data help identify shows customer want to watch by informing the employee, but it also helps them run a better system.  Netflix would be nothing without the data they are able to obtain for their ecosystems of micro services.  The essential lifeblood of their operation platform.  With the use of big data they can fin tune how they run their micro services and help maintain the four nines of uptime the hold themselves to.  With this data, they spot what services are essential for the product to still run, though it may be limited.  They also use a lot for testing suits for different types of failures in production under load.  These then allow them to get data back on how the system performed and how to fix it without actually breaking anything at all.
\section{Conclusion}
In conclusion, this is only a small view of what makes Netflix a Big Data giant.  They have faced many other issues in the past and over come them with agility.  Normally if a company grew this fast it would start to get really messy and make some possibly bad decisions based of their data.  Just ask UBER about that type of situation.  However, that issue may have been to lack of leadership and planning more than anything.  In any case Netflix has adapted and conquered and now is a model for many other platforms out there such as amazon video and apple TV.  Whether or not the will get to the level of Netflix is possible, but the real question is, by the time they get to that level, where will Netflix be?\cite{Evans}
\begin{acks}
  The authors would like to thank Dr. Gregor von Laszewski and the TA's for their help and support.
\end{acks}
\bibliographystyle{ACM-Reference-Format}
\bibliography{report} 
\end{document}
