\documentclass[sigconf]{acmart}
\usepackage{graphicx}
\usepackage{hyperref}
\usepackage{todonotes}

\usepackage{endfloat}
\renewcommand{\efloatseparator}{\mbox{}} % no new page between figures

\usepackage{booktabs} % For formal tables

\settopmatter{printacmref=false} % Removes citation information below abstract
\renewcommand\footnotetextcopyrightpermission[1]{} % removes footnote with conference information in first column
\pagestyle{plain} % removes running headers

\newcommand{\TODO}[1]{\todo[inline]{#1}}
\begin{document}
\title{Big Data in Oceanography}
\author{Zachary Meier}
\orcid{1234-5678-9012}
\affiliation{%
  \institution{Indiana University}
  \city{Bloomington} 
  \state{Indiana} 
  \postcode{47408}
}
\email{zrmeier@indiana.edu}
% The default list of authors is too long for headers}
\renewcommand{\shortauthors}{Z. Meier et al.}
\begin{abstract}

  This paper will give an overview of how big data is used to collect
  and process data about our oceans in better ways to understand it
  and how to solve complex issues surrounding it.
 
\end{abstract}
\keywords{hid346, i513}
\maketitle
\section{Introduction}

Ever since man as first seen the ocean, it has always had an air of
mystery to it.  Even now in the age of information, we still barely
know anything about it.  This paper is to give an overview of how and
with what we are collecting the data and the problems we face in that
collection of data.  In the hope that we can start to uncover the
mysteries that hide below the surface. To collect this data there is a
use of many different sensors. \cite{NOAA}

\begin{description}

\item [Acoustic Doppler Current Profiler:] This tool measures the speed
  and direction of ocean currents using the principle of Doppler
  shift. Measuring currents is a fundamental practice of physical
  oceanographers.

\item [Technologies for Ocean Acoustic Monitoring:] This technology
  listens to the ocean for all sounds, boats, sea animals, waves,
  siezmic activity.

\item [The Bushmaster and the Chimneymaster:] A collection net used to
  grab tube worms or living fauna near geothermal vents.  These tools
  are typically attached to a submersible vessel.

\item [Clod Cards:] These plaster cards track the motion of water for
  benthic organisms.  The organisms that inhabit the bottom of the
  ocean.  allowing us to learn more about the harder to reach parts of
  the ocean.

\item [Drifters:] Drifters, are essentially devices that flow with the
  current of the ocean.  Allowing for them to be mapped and
  visualized.

\item [Mapping: Geographic Information Systems:] Essentially the
  creation of 3D modeling within an computer environment of the ocean.

\item [Satellites:] Can detect and observe the ocean characteristics.

\item [Semipermeable Membrane Devices:] Used to collect various
  microbes for analysis of bacteria environment.

\item [Sonar:] Uses sound to detect area around a submersible, and
  well as figure out water depth.

\item [Sonde and CTD:] Collect data on a multitude of things,
  primarily temperature at different depths and conductivity of the
  water.

\end{description}

These sensors and technologies are what help make the data we can use
to analyze and help predict our oceans health and patterns.  All of
these tools collect different data, from the directions of currents,
temperatures, bacteria, as well as mapping data.  With all of this
information, it is now becoming a challenge to ask the right questions
of it.  That is the main issue in the field right now is about asking
the right questions.  With the oceans being the least explored and
mysterious place on earth we are very unsure how the system works as a
whole and the cause and effect major events have on it.

\section{Problems the Ocean Faces}

The ocean, is a big an complex ecosystem, no only that, but is rather
sensitive to change.  Given humanities treatment of the ocean and it
inhabitant, it has taken quite a few damaging blows to say the least.
As of now there are multiple threats to the ocean.  There is the
climbing temperatures, which effect the coral reefs and ocean level.
The ever present amount of plastic in the ocean.  From large vortex
patches in the middle of the oceans that are miles wide, to the small
micro plastics fish are eating, and intern are being eaten by us. Not
to mention the catastrophic effects of the multiple oil spills, and
bleaching incidents that are destroying the coral reef habitats.
Which coral reefs have two fold purpose, one is the habitat, the
second is the reef itself which helps remove a lot of carbon dioxide
from the atmosphere.  These are the most pressing issues as of right
now but there are many more.  With the use of big data and collection
through the sensor above we can start learning how to take action and
where.

\subsection{Collection Efforts}

As of right now the collection of data is not a collective effort
globally. \citep{Delgnrain} However there are many organizations that
are doing their best to collect and analyze the data.  One of these
organizations is The National Oceanic and Atmospheric Association.
Which is collecting massive amounts of off coastal based laboratories.
The use of data is to help get an unbiased amount of data to help
predict the health of the ocean and its effect on climate change and
rising sea levels. That way it can help coastal cities and states,
plan for the future within the next couple decades.

Even with organizations like this around the globe, there is still not
enough data to truly predict and know the health of our oceans.  It
will take a global effort and many more sensors, and a lot more
interest from the public.  This will help with funding for the
scientist and the tools that will need to be developed and maintained
for future exploration.

With enough data we could finally see how our planets oceans work.  It
could help choose the best route for boats, based off currents.  It
could help prevent major disasters by helping cities, states,
countries be prepared for hurricanes weeks in advanced.  It would
allow us to see how events in different parts of the globe would have
an effect elsewhere.

\section{The Data}

As with any situation where massive amounts of data are involved,
things can get bit complicated.  In the case of ocean data, you have
data from multiple sources.  Ranging from satellites, to buoys, to
bacterial analysis.  Not only that but the complexity with all of this
data is that the ocean is one of the most dynamic things on earth.
Unlike things on land, though always moving, are usually predictable.
The ocean does not give any such luxury as it is all connected and
always in motion.  Given this fact, it is hard to collect data for the
ocean as a whole.  At least for right now.  However, it is possible to
get fairly reliable data for a set area.

\subsection{Integration}

Making sense of all this data is the main issue that most scientist
face.  So integration is a primary concern.  When modeling the data
especially there is a lot of work to make the data come together and
makes sense.  In one case for modeling salinity of different parts of
coastal regions, they match up GPS data points and observations and
time with the data from the senors that collect it.  The matching of
all this information allows you to integrate the data together for
correct mapping of a given location. \cite{Liu2016} This is just a
small example of the amount of effort to pull all the data together to
produce visualization of those results


\section{Solutions}

Though the efforts have not come together globally as of yet, plans to
make that happen are in motion. Just this June, the United Nations met
for a week long discussion on how to make our oceans healthier with
emphasis on global cooperation. \cite{Woody} Unlike the pairs
agreements for global warming, this meeting will not come to a signed
contract, rather an understanding that be every country will to be
accountable for some part of the oceans health.  And encourage
international cooperation to help solve our biggest issues.

\subsection{Hopeful Future}

At this point there is no clear cut solution to solve all issues that
face our oceans.  And we may never collect all the data that we need
to make the most informed decision.  However with the data collected
thus far is clear that we need to do. And though we do not have it all
the answers, we have enough data to start asking the important
questions to set us on the right track.
 
\begin{acks}

  The authors would like to thank Professor Gregor von Laszewski and
  all the TA's for their help.  As well as other students and their
  contribution to collective learning.

\end{acks}
 
\bibliographystyle{ACM-Reference-Format}
\bibliography{report} 
 
\end{document}
